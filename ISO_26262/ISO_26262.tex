% ToDo:
% - Seiten cite mit "Seite" oder "S."

\documentclass[a4paper,DIV=calc,ngerman]{scrartcl}

\usepackage[utf8]{inputenc}
\usepackage[T1]{fontenc}
\usepackage[ngerman]{babel}
\usepackage{setspace}
\usepackage{microtype}
\usepackage{lmodern}
\usepackage[pdftex]{graphicx}
\graphicspath{{images/}}
\usepackage{wrapfig}
\usepackage[
  pdftex,
  bookmarks, bookmarksopen, bookmarksopenlevel=1, bookmarksnumbered=true,
  pdfpagemode={UseNone},
  pdfpagelayout={SinglePage},
  plainpages=false,
  pdfkeywords={Robuste Systeme, ISO 26262},
  pdfsubject={Robuste Systeme - ISO 26262},
  pdftitle={Robuste Systeme - ISO 26262},
  pdfauthor={Martin Wichmann}
]{hyperref}
\usepackage{booktabs}
\usepackage{colortbl}
\usepackage{multirow}

\definecolor{hellgrau}{rgb}{0.95,0.95,0.95}


\begin{document}


\titlehead{\center{\large \textsc{Ostfalia Hochschule für angewandte Wissenschaften}}}
\title{Robuste Systeme - ISO 26262}
\author{Martin Wichmann\\701\,277\,37}
\date{\today}
\maketitle

\tableofcontents


%%%%%%%%%%%%%%%%%%%%%%%%%%%%%%%%%%%%%%%%%%%%%%%%%%%%%%%%%%%
% Was ist die ISO
% DIN 61508...
% TODO: was ist funktionale Sicherheit
\section{Einleitung}
\label{sec:Einleitung}
Diese Ausarbeitung beschäftigt sich mit der Norm ISO/DIS 26262. Dabei handelt es sich um eine neue Norm für Funktionale Sicherheit die gezielt für den Automobil Bereich entwickelt wurde. Bereits seit 1998 existiert die DIN EN 61508 unter dem Namen "`Funktionale Sicherheit sicherheitsbezogener elektrischer/elektronischer/programmierbarer elektronischer Systeme"'. Diese ist jedoch im Bezug zum Automobil Bereich zum Teil unklar formuliert. So zum Beispiel endet der Produkt-Zyklus bei der DIN 61508 mit dem Aufstellen des Produktes. Diese Vorgehensweise ist zwar zum Beispiel bei Chemie-Anlagen korrekt, jedoch im Automobil Bereich nicht vollständig. Hier folgt auf die Fertigstellung des Produktes die Serien-Produktion, der Betrieb, der Kundendienst und zum Schluss die Außerbetriebnahme. Aus diesem Grund wurde 2011 die ISO/DIS 26262 freigegeben. Diese Norm deckt genau die erwähnten Punkte ab und spezialisiert sich unter anderem im Bereich Hardware und Software. Dazu gehören auch einige Beispiele die auf den Automobil Bereich zugeschnitten sind und so direkt Anwendung finden können.

% TODO: sind es "keine" oder "kaum" Zitate aus der Norm???
Da die ISO 26262 der Hochschule nicht zur Verfügung steht und es aufgrund des aktuellen Veröffentlichkeiskeitsdatums des Standards nur wenig Literatur dazu existiert, basiert diese Ausarbeitung vor allem auf \cite{1}. Dieses Buch thematisiert vor allem den DIN EN 61508, betrachtet jedoch auch in einem recht Ausführlichen Kapitel den ISO 26262 Standard. Aus diesem Grund werden die Informationen im weiteren vor allem aus dem genannten Buch stammen und es werden keine direkten Zitate aus der Norm verwendet.

Die Struktur dieser Ausarbeitung orientiert sich grob an der Struktur des Standards und des Sicherheitslebenszyklus. Dabei wird vor allem Wert auf einige konkrete Punkte gelegt.

%%%%%%%%%%%%%%%%%%%%%%%%%%%%%%%%%%%%%%%%%%%%%%%%%%%%%%%%%%%
% Aufbau der ISO
\section{Struktur und Überblick ISO 26262}
\label{sec:Struktur}
Die ISO 26262 ist in insgesamt 10 Abschnitte aufgeteilt. Dabei behandelt jedes Kapitel einen bestimmten Themenbereich der funktionalen Sicherheit.

\begin{enumerate}
    \item Vokabular
    \item Management der funktionalen Sicherheit
    \item Konzeptphase
    \item Produktentwicklung: Systemebene
    \item Produktentwicklung: Hardwareebene
    \item Produktentwicklung: Softwareebene
    \item Produktion, Betrieb und Außerbetriebnahme
    \item Unterstützende Prozesse
    \item ASIL- und sicherheitsorientierte Analysen
    \item Guideline
\end{enumerate}

Abschnitt 1 führt das in der Norm verwendete Vokabular ein. In Abschnitt 2 werden Organisatorische Konzepte eingeführt die verwendet werden um die funktionale Sicherheit während des gesamten Projektes zu gewährleisten.

Entsprechend des Titels befasst sich Abschnitt 3 mit der Konzeptphase. In diesem Zusammenhang ist vor allem die "`Gefährdungsanalyse und Risikoabschätzung"' zu nennen. Hierin werden alle potentiellen Gefährdungen des Systems identifiziert und mögliche Konsequenzen ausgewertet. Anschließend wird jeder Gefährdung entsprechend der Vorgaben ein Sicherheitslevel (ISO 26262: ASIL, DIN 61508: SIL) zugeordnet. Dieses gibt an in wie weit diese Gefährdung ein Sicherheitsrisiko darstellt.

Abschnitte 4 bis 6 befassen sich mit der eigentlichen Entwicklung des Produktes. Hier werden Vorgehensweisen wie zum Beispiele V-Modelle definiert. Außerdem werden entsprechend dem ASIL Anforderungen und Methoden definiert die eingehalten werden müssen.

Die restlichen Abschnitten befassen sich jeweils mit den titelgebenden Bereichen. Dabei hervorzuheben sind vor allem Abschnitte 8 und 10. Abschnitt 8 erläutert zum Beispiel Schnittstellen zwischen verteilten Entwicklungs-Partnerschaften, Konfigurations-Management, Dokumentation und Qualifizierung von Software-Werkzeugen und -Komponenten. Im Abschnitt 10 werden vor allem Beispiele und weiterführende Informationen zum Standard gegeben.


%%%%%%%%%%%%%%%%%%%%%%%%%%%%%%%%%%%%%%%%%%%%%%%%%%%%%%%%%%%
\section{Sicherheitslebenszyklus}
\label{sec:Sicherheitslebenszyklus}
Der Sicherheitslebenszyklus, im weiteren kurz Lebenszyklus genannt, umfasst eine Reihe von Punkten von der Konzeptphase bis hin zur Produktions- und Servicephase. Hierbei werden Anforderung an jeden Teilpunkt gestellt, mit dem Ziel funktionale Sicherheit zu gewährleisten.

% TODO: Abbildung 6-1

Abbildung ??? zeigt den Lebenszyklus nach ISO 26262. Dieser stellt eine konkrete Interpretation des Lebenszyklus der DIN 61508 dar. Die Nummerierungen innerhalb der Abbildung geben an, in welchem Kapitel des Standards die konkreten Regelungen zu finden sind. 

Die Punkte "`Item definition"' und "`Initiation safety lifecycle"' werden benötigt um das geplante Produkt zu beschreiben. Hier werden alle Eigenschaften des Produktes festgelegt und Anforderungen definiert. Im nächsten Punkt "`Hazard analysis and risk assessment"' werden alle möglichen Gefahren udn deren Folgen betrachtet. Aus diesen wird im Punkt "`Functional safety concept"' ein Sicherheitsplan entwickelt der den kompletten Lebenszyklus abdeckt und die Konzept-Phase abschließt. 

Die Produkt-Entwicklungs-Phase ist relativ selbst erklärend. Hier wird das Produkt nach den Vorgaben des Sicherheitsplans entworfen. Zusätzlich werden die Produktion, Wartung und Außerbetriebnahme geplant.

In der letzten Phase "`After SOP"' wird das Produkt schlussendlich produziert und ausgeliefert. Obwohl der Sicherheitsplan vor allem zu Beginn entwickelt wird, werden während des kompletten Lebenszyklus Erweiterungen und Verbesserungen vorgenommen. Dabei ist es wichtig jederzeit eine Nachvollziehbarkeit zu gewährleisten. Es sollte also jederzeit klar sein wer, wann, warum, welche Änderungen vorgenommen hat. Außerdem können während aller Phasen des Projektes Reviews, Audits und Sicherheitsassesments durchgeführt werden. \cite[S. 121]{1}





%%%%%%%%%%%%%%%%%%%%%%%%%%%%%%%%%%%%%%%%%%%%%%%%%%%%%%%%%%%
% Angepasste Prozesse
% Ausbildungsnachweise
% Sicherheitsplan... Seite 121
\section{Management der funktionalen Sicherheit}
\label{sec:Management}
Das Management der funktionalen Sicherheit ist der Konzept-Phase zuzuordnen. In diesem Zusammenhang werden unter anderem folgende Dokumente gefordert (vgl. \cite[S. 121]{1}):

\begin{itemize}
    \item Ausbildungs- und Qualifizierungsnachweise
    \item Sicherheitsplan
    \item Projektplan/Projekthandbuch inklusive der Planung der sicherheitsbezogenen Vorgänge
    \item Review-, Audit- und Assessmentplan
    \item Sicherheitsnachweise
\end{itemize}

% TODO: etwas darüber schreiben...




%%%%%%%%%%%%%%%%%%%%%%%%%%%%%%%%%%%%%%%%%%%%%%%%%%%%%%%%%%%
% Gefährdung E,S,C -> ASIL
% Seite 122
\section{Gefährdungsanalyse und Risikoeinschätzung}
\label{sec:GundR}
Ein Kernpunkt der Konzept-Phase ist die Gefährdungsanalyse und Risikoeinschätzung, im weiteren kurz G\&R. Im ISO Standard 26262 ist eine Vorgehensweise beschrieben, mit der verschiedene Gefahren analysiert und in Klassen eingeteilt werden. Dabei wird folgender Ablauf vorgegeben (vgl. \cite[S. 123]{1}):

\begin{itemize}
    \item Ermitteln aller relevanten Fahrzeugzustände und Fahrsituationen
    \item Ermitteln möglicher funktionaler Fehler
    \item Bewerten der Risiken jeder Gefährdungssituation in allen Fahrsituationen
    \item Festlegen der notwendigen Risikominderung (ASIL)
    \item Ableiten der Sicherheitsziele
\end{itemize}

Im ersten Schritt wird eine Liste von möglicher Zuständen erstellt. Dazu gehören zum Beispiel Autobahnfahrt, Landstraße, stop-and-go. Im Anhang des Standards sind eine Reihe von Beispielen dazu aufgeführt die als Anhaltspunkt dienen können. Im nächsten Schritt wird ermittelt welche funktionalen Fehler auftreten können. Hierzu werden die zu Beginn definierten Informationen und Experten verwendet um einen Katalog von möglichen Fehlern zu ermitteln. Beispiel hierzu sind:

\begin{itemize}
    \item Fahrlicht schaltet unmotiviert ab
    \item Fahrlicht schaltet unmotiviert ein
    \item Fahrlicht flackert
\end{itemize}

Nachdem nun sowohl Fahrbedingungen und mögliche Fehler aufgelistet wurden, werden nun die möglichen Risiken betrachtet. Hierzu werden alle Zustände mit allen Risiken betrachtet und nach einem vorgegebenen Schema bewertet. Dieses Schema ist wie folgt aufgebaut:

\begin{itemize}
    \item Häufigkeit der Fahrsituation (Exposure E)
    \begin{itemize}
        \item E0: Unvorstellbar
        \item E1: Sehr niedrige Wahrscheinlichkeit
        \item E2: Niedrige Wahrscheinlichkeit
        \item E3: Mittlere Wahrscheinlichkeit
        \item E4: Hohe Wahrscheinlichkeit
    \end{itemize}
    \item Schwere eines möglichen Schadens (Severity S)
    \begin{itemize}
        \item S0: Keine Verletzungen
        \item S1: Leichte und mittlere Verletzungen
        \item S2: Schwere Verletzungen - Überleben wahrscheinlich
        \item S3: Lebensgefährliche Verletzungen - Überleben unwahrscheinlich
    \end{itemize}
    \item Beherschbarkeit druch den Fahrer (Controllability C)
    \begin{itemize}
        \item C0: Im Allgemeinen beherschbar
        \item C1: Einfach beherschbar
        \item C2: Normalerweise beherschbar
        \item C3: Schwierig oder nicht beherschbar
    \end{itemize}
\end{itemize}

In Schritt vier wird nun aus den Risikoabschätzungen ein Automotive-Sicherheit-Intergritätslevel (ASIL) abgeleitet. Dieses kann die Werte QM, A, B, C und D annehmen. Hierbei gibt QM an, dass keine Maßnahmen zur Risikominderung nötig sind. ASIL A ist die niedrigste Klasse und D die höchste. Diese Einteilung wirkt sich direkt auf die geforderten Sicherheits-Maßnahmen aus. Es wird jeweils der höchste vorhandene ASIL für jede Fahrsituation gewählt. Der ASIL wird nach folgendem Schema gewählt.


\begin{table}[h]
\center
\begin{tabular}[h]{c c | c c c}
\toprule
 &  & C1 & C2 & C3\\
\midrule
\multirow{4}{*}{S1} & E1 & QM & QM & QM\\
 & E2 & QM & QM & QM\\
 & E3 & QM & QM & ASIL A\\
 & E4 & QM & ASIL A & ASIL B\\
\midrule
\multirow{4}{*}{S2} & E1 & QM & QM & QM\\
 & E2 & QM & QM & ASIL A\\
 & E3 & QM & ASIL A & ASIL B\\
 & E4 & ASIL A & ASIL B & ASIL C\\
\midrule
\multirow{4}{*}{S3} & E1 & QM & QM & ASIL A\\
 & E2 & QM & ASIL A & ASIL B\\
 & E3 & ASIL A & ASIL B & ASIL C\\
 & E4 & ASIL B & ASIL C & ASIL D\\
\bottomrule
\end{tabular}
\caption{Einteilung der ASIL aufgrund der Risikoeinschätzung vgl. \cite[S. 125]{1}}
\label{tab:asil}
\end{table}

Der letzte Schritt besteht nun aus dem ableiten von angemessenen Sicherheitszielen. Diese sind den Gefährdungen und Konsequenzen entsprechend zu wählen und im Sicherheitsplan zu beschreiben.

Da die ASIL Level prinzipiell eine subjektive Einschätzung sind, gibt der Standard einige Beispiel (vgl. \cite[S. 125]{1}) anhand derer die Gefährdungen eingeteilt werden können.

Beispiel für Häufigkeit der Fahrsituation (Exposure E):
\begin{itemize}
    \item E4 = bei jeder Fahrt (Beispiel: Schalten, Beschleunigen)
    \item E2 = mehrmals im Jahr (Beispiel: Anhängerfahrt, Dachgepäckträger)
\end{itemize}

Beispiel für für die Schwere eines Unfalls (Severity S):
\begin{itemize}
    \item S3 = Seitenaufprall eines anderen Fahrzeugs mit $ < 35 km/h $
    \item S0 = Anfahren an einen Zaun oder Negrenzungspfahl mit $ \Delta v < 15 km/h $
    \item S1 = Frontalzusammenstoß von zwei PKWs mit $ \Delta v < 20 km/h $
\end{itemize}

Beispiel für Beherrschbarkeit (Controllability C):
\begin{itemize}
    \item C2 = Fahrer kann Farhzeug normalerweise bei Dunkeheit auf einer unbeleuchteten Landstraße bei totalem Lichtausfall stoppen, ohne von der Fahrbahn zu kommen
    \item C1 = Fahrer kann in der Regel Personenschäden durch Bremsen vermeiden, wenn die Lenksäule beim Anfahren blockiert
\end{itemize}

Als Beispiel für eine gesamte Analyse wird eine Servolenkung betrachtet (vgl. \cite[S. 217]{1}). Hierbei werden zwei mögliche Szenarien ausgewählt. Eine vollständige Analyse würde weitaus umfangreicher ausfallen müssen.

Erstes Szenario: Bei einer Stadtfahrt (etwa 50 km/h) kommt es zu unmotivierten Lenkbewegungen.
\begin{itemize}
    \item E = E4
    \item C = C3
    \item S = S2
\end{itemize}

Dieses Szenario würde zu einem ASIL C führen. Die selbe Analyse bei einer Landstraßenfahrt (etwa 100 km/h) führt zu ASIL D. Damit wird diese Gefährdung als ASIL D bewertet.

Zweites Szenario: Ausfall der Lenkkraftunterstützung bei einer langsamen Fahrt ( $ <= 20 km/h $ ).

\begin{itemize}
    \item E = E4
    \item C = C2
    \item S = S1
\end{itemize}

Dieses Szenario führt zu einem ASIL A. Jedoch führt die selbe Situation bei einer höheren Geschwindigkeit zu einem ASIL B. Diese Gefährdung wird also mit ASIL B bewertet.

%%%%%%%%%%%%%%%%%%%%%%%%%%%%%%%%%%%%%%%%%%%%%%%%%%%%%%%%%%%
% Abb. 6-3
% Seite 129
\section{Systemebene}
\label{sec:Systemebene}
asd




%%%%%%%%%%%%%%%%%%%%%%%%%%%%%%%%%%%%%%%%%%%%%%%%%%%%%%%%%%%
% Obergrenze Ausfall bei ASIC
% Fehlerarten
% FMEA, FTA
% Seite 133
\section{Hardware}
\label{sec:Hardware}
asd




%%%%%%%%%%%%%%%%%%%%%%%%%%%%%%%%%%%%%%%%%%%%%%%%%%%%%%%%%%%
% V-Modell Abb 6-10
% Metriken
% Arbeitsprodukte
\section{Software}
\label{sec:Software}
asd




%%%%%%%%%%%%%%%%%%%%%%%%%%%%%%%%%%%%%%%%%%%%%%%%%%%%%%%%%%%
% Seite 149
\section{Produktion und Betrieb}
\label{sec:}
asd





%%%%%%%%%%%%%%%%%%%%%%%%%%%%%%%%%%%%%%%%%%%%%%%%%%%%%%%%%%%
% Was haben wir aus dem ganzen Kram gelernt ;-)
\section{Fazit}
\label{sec:Fazit}
asd




%%%%%%%%%%%%%%%%%%%%%%%%%%%%%%%%%%%%%%%%%%%%%%%%%%%%%%%%%%%
% Ein paar Quellenangaben...
\begin{thebibliography}{------}
\label{sec:Literatur}

\bibitem[1]{1} \textsc{Peter Löw, Roland Pabst, Erwin Petry}: {\em Funktionale Sicherheit in der Praxis.} dpunkt.verlag, 2010.

\end{thebibliography}





\end{document}
