%% 
%% Master Thesis
%% Autor: Martin Wichmann
%% 
%% Todo:
%% 


\documentclass[
  a4paper,					    % a4paper (sic!)
  %BCOR10mm,				    % Korrektur des innneren Randes bei Bindung (Bindungskorrektur)
  %DIV12,					    % je groesser die Zahl, desto kleiner der Rand
  %11pt,        			    % Schriftgroesse
  %oneside,
  twoside,
  %openright,   				% doppelseitig und jedes Kapitel faengt auf der rechten Seite an
  %pagesize,					% schreibt die Seitengroesseninformationen in die PDF Datei
  DIV=calc,     				% KOMA-Script soll den optimalen Satzspiegel berechnen
  %DIVclassic, 					% mittelalterlicher Buchseitenkanon
  %headsepline,					% Trennlinie
  %footsepline,					% Trennlinie
  %headtopline,					% Trennlinie
  %footbotline,					% Trennlinie
  %noonelinecaption,			% setzt die Bildueberschrift unabhaengig von der Laenge immer linksbuendig
  %liststotoc,
  %idxtotoc,
  bibliography=totoc,
  %bibtotocnumbered, liststotocnumbered,
  %liststotoc, idxtotoc, bibtotoc,
  %tablecaptionbelow,
  %tablecaptionabove,	        % Tabellenbeschriftung unter oder oben
  %abstracton,  				% Überschrift der Zusammenfassung aktivieren
  %chapterprefix,				% "Kapitel xx" vor jeder Kapitelueberschrift
  %cleardoublestandard,
  %cleardoubleplain,
  cleardoublepage=empty,
  %smallheadings,
  %normalheadings,
  %makeidx,
  ngerman,     					% allen Paketen die Hauptsprache mitteilen
  %draft       					% draft version  
  final       					% final version
]{scrbook}	
%%%%%%%%%%%%%%% Grundlegendes %%%%%%%%%%%%%%%
\usepackage[utf8]{inputenc}
\usepackage[T1]{fontenc}
\usepackage[english,ngerman]{babel} 	% Unterstuezung fuer englisch und deutsch
\usepackage{setspace}			% erlaubt 1 1/2 fachen Zeilenabstand
%%%%%%%%%%%%%%% Tabellen %%%%%%%%%%%%%%%
%\usepackage{array}
%\usepackage{tabularx}
\usepackage{booktabs}			% \toprule, \midrule und \bottomrule in Tabellen
%\usepackage{multirow}
%%%%%%%%%%%%%%% Mathe %%%%%%%%%%%%%%%
%\usepackage{amsmath, amsthm, amscd, amssymb, amsfonts}
%\usepackage{ziffer}
%\usepackage{icomma}
%%%%%%%%%%%%%%% Typografie %%%%%%%%%%%%%%%
%\usepackage{mparhack}			% workaround for a LaTeX bug in marginpars
\usepackage{ellipsis}			% fix uneven spacing around ellipses in LaTeX text mode.
\usepackage{microtype} 			% optischer Randausgleich (font expansion and character protrusion)
%%%%%%%%%%%%%%% Schriften %%%%%%%%%%%%%%%
\usepackage{lmodern}
%\usepackage{textcomp} 			% Sonderzeichen
%\usepackage{mathcomp}
%\usepackage{chemsym}
%%%%%%%%%%%%%%% Sonstiges %%%%%%%%%%%%%%%
%\usepackage{url}
\usepackage{scrhack}            % KOMA Paket um zussammenarbeit mit anderen Paket (listings) zu verbessern
\usepackage{listings}			% Code-Abschnitt mit Syntax-Highlighting
\lstset{
%language=C,
breaklines=true,
breakatwhitespace=true
basicstyle=\footnotesize,
numbers=left,
numberstyle=\footnotesize,
stepnumber=2,
numbersep=5pt,
extendedchars=true,
inputencoding=utf8,
breakindent=30pt,
escapeinside={\%(}{\%)},
%xleftmargin=20pt				% Einrückung der listings
}
%\usepackage{color} 			% Farben
%%%%%%%%%%%%%%% Grafiken %%%%%%%%%%%%%%%
\usepackage[pdftex]{graphicx}	% das pdftex soll das Handling von Bildern verbessern
\graphicspath{{images/}}		% Bilder im Verzeichnis images suchen
\usepackage{wrapfig}
%\usepackage[hang]{subfigure}	% Subabbildungen
%\usepackage{tikz}				% zum Zeichnen (Frontend zu PGF)
%\usepackage{rotating} 			% fuer gedrehte Tabellen und Bilder
%\usepackage{pdfpages}
\usepackage[
  pdftex,
  bookmarks, bookmarksopen, bookmarksopenlevel=1, bookmarksnumbered=true,
  pdfpagemode={UseNone},		% UseNone, FullScreen, UseThumbs, UseOutlines, (UseOC, UseAttachments)
  pdfpagelayout={TwoPageRight},	% SinglePage, OneColumn, TwoColumnLeft, TwoColumnRight, TwoPageLeft, TwoPageRight
  plainpages=false, 
  pdfkeywords={},
  pdfsubject={},
  pdftitle={Master-Arbeit},
  pdfauthor={Martin Wichmann}
]{hyperref}
%%%%%%%%%%%%%%% Zitieren und Index %%%%%%%%%%%%%%%
% Ich habe im Internet gelesen, dass cite nach hyperref stehen soll?!
\usepackage[numbers]{natbib}
% Index:
%\usepackage{makeidx}					% Paket für die Indexerstellung
%\makeindex
% Glossar:
%\usepackage[style=super, header=none, border=none, number=none, cols=2, toc=true]{glossary}
%\usepackage[style=altlist, number=none]{glossary}
%\makeglossary
% Benutzung des Glossars: \glossary{name={Schnauze}, description={Fachausdruck für die Hundenase.},}
%                         \glossary{name={Knochen}, description={Lieblingsspeise eines jeden Hundes.},}
%%%%%%%%%%%%%%%%%%%%%%%%%%%%%%%%%%%%%%%%%%%%%%%%%%%%

%%%%%%%%%%%%%%%%%%%%%%%% Eigene Definitionen %%%%%%%%%%%%%%%%%%%%%%%%%%%%%%

%%% Bei Verwendung von BibTex: %%%%%%%%%%%%%%%%%%%%
\addto{\captionsngerman}{\renewcommand*{\bibname}{Quellenverzeichnis}}
% sonst
%\renewcommand{\bibname}{Quellenverzeichnis}

%%% Trennungen %%%%%%%%%%%%%%%%%%%%%%%%%%%%%%%%%%%%%%%%%%%
\hyphenation{}

%%% 1 1/2 fachen Zeilenabstand wählen %%%%%%%%%%%%%%%%%%%%
\onehalfspacing
\typearea[current]{calc}				% Neuberechnung des Satzspiegels

%%%%%%%%%%%%%%%%%%%%%%%%%%%% Begin document %%%%%%%%%%%%%%%%%%%%%%%%%%%%%%%
\begin{document}
\selectlanguage{ngerman}
\frontmatter

%%%%%%%%%%%%%%%%%%%%%%%%%%%%%%%%% Titel %%%%%%%%%%%%%%%%%%%%%%%%%%%%%%%%%%
\titlehead{\center{\large \textsc{Ostfalia Hochschule für angewandte Wissenschaften}}}
\subject{Master-Arbeit}
\title{}
\author{Martin Wichmann\\Matrikel ???\,???\,??}
\date{Eingereicht am TODO}
\publishers{Prüfer: 

  TODO

  TOOD}

%\uppertitleback{%
%  Beteiligte Institutionen:\\
%  \parbox{0.5\textwidth}{%
%  \centering \includegraphics[width=3.5cm]{logofh} \\
%  Fakultät Informatik \\ Ostfalia Hochschule für angewandte Wissenschaften } \hfill
%}

\lowertitleback{Diese Arbeit wurde mit Hilfe von Freier Software erstellt: \\
Gesetzt mit Hilfe von {\KOMAScript} und {\LaTeX}. LibreOffice für die \\ Textverarbeitung und gedit als Editor. Xubuntu als offenes Betriebssystem. \\ \\ Diese Arbeit ist freigegeben unter der Creative Commons CC-BY-SA Lizenz\cite{CCBYSA}.
}
\dedication{TODO\\ \vspace{1cm}
\textit{TODO}
}

\begin{singlespace}
\maketitle

%%%%%%%%%%%%%%%% Abstract %%%%%%%%%%%%%%%%%%%%%%%


\section*{Zusammenfassung}
\pdfbookmark[1]{Zusammenfassung}{Zussammenfassung}
TODO
\vfill

\foreignlanguage{english}{
\section*{Abstract}
TODO
\vfill
}


\thispagestyle{empty}

%%%%%%%%%%%%%%%%%%%%%%%%%%%%%% Verzeichnisse %%%%%%%%%%%%%%%%%%%%%%%%%%%%%%

\tableofcontents               	% Inhaltsverzeichnis
%\listoffigures               		% Abbildungsverzeichnis
%\listoftables             			% Tabellenverzeichnis
%\lstlistoflistings          		% Listenverzeichnis
%\listoflistings					% Quellcodeverzeichis
%\printglossary               		% Formelverzeichnis
\end{singlespace}


%%%%%%%%%%%%%%%%%%%%%%%%%%%%%% Zeilenabstand %%%%%%%%%%%%%%%%%%%%%%%%%%%%%%
% Das Setzen eines anderen Abstandes mitten im Dokument kann zu Fehlern führen (vgl. scrguide S. 30)
%\doublespacing
%\onehalfspacing
%\typearea[current]{last}					% stammt aus scrguide S. 30
%\typearea[current]{calc}					% Neuberechnung des Satzspiegels









%%%%%%%%%%%%%%%%%%%%%%%%%%%%%%% Einleitung %%%%%%%%%%%%%%%%%%%%%%%%%%%%%%%%%%
\mainmatter
\chapter{Einleitung}
\label{sec:Einleitung}
TODO




%%%%%%%%%%%%%%%%%%%%%%%%%%%%%%% Zusammenfassung %%%%%%%%%%%%%%%%%%%%%%%%%%%%%%%%%%
\chapter{Fazit und Ausblick}
\label{sec:FazitAusblick}
TODO




%%%%%%%%%%%%%%%%%%%%%%%%%%%%%%%%%%% Anhang %%%%%%%%%%%%%%%%%%%%%%%%%%%%%%%%%
\backmatter
\appendix
\part*{Anhang}
TODO





%%%%%%%%%%%%%%%%%%%%%%% Erklärung und CD %%%%%%%%%%%%%%%%%%%%%%%%%%%%%
\chapter{Erklärung}
\label{sec:Erklärung}
Hiermit erkläre ich, dass ich die vorliegende Arbeit selbstständig und nur unter Verwendung der angegebenen Quellen und Hilfsmittel erstellt habe.
\vspace{2.5cm} \par
Wolfenbüttel, den TODO


\chapter{Beigelegte CD}
\label{sec:BeigelegteCD}
TODO



%%%%%%%%%%%%%%%%%%%%%%%%%%%%%%% Bibliographie %%%%%%%%%%%%%%%%%%%%%%%%%%%%%%
% \setbibpreamble{Die Quellenangaben sind alphabetisch nach den Namen der Autoren sortiert.
% Bei mehreren Autoren wird nach dem ersten Autor sortiert.\par\bigskip\bigskip}
%
% Quellen, die nicht direkt zitiert wurden, aber trotzdem hier erscheinen sollen!
%\nocite{tbinformatik}\nocite{tb_et2000}\nocite{tb_mathe1999}
\nocite{*}
%
\begin{singlespace}
\bibliographystyle{natdin}	% alphadin, plaindin, abbrvdin, unsrtdin, natdin
					            % germbib: gerabbrv, geralpha, gerplain, gerunsrt, gerapali, gerxampl
                           		% plainnat, abbrvnat, unsrtnat
                           		% Am besten geeignet: natdin
                           		% oder was ist mit dinat
                           		% ???apalike???
\bibliography{literature}
\end{singlespace}
%%%%%%%%%%%%%%%%%%%%%%%%%%%%%%%%%%%%%%%%%%%%%%%%%%%%%%%%%%%%%%%%%%%%%%%%%%%%

\end{document}


