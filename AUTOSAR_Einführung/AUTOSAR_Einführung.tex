% TODO:
% - 

%%%%%%%%%%%%%%%%%%%%%%%%%%%%%%%%%%%%%%%%%%%%%%%%%%%%
% documentclss
\documentclass[]{beamer}
%\documentclass[handout]{beamer} %Drucker Version
%\documentclass[draft]{beamer}


%%%%%%%%%%%%%%%%%%%%%%%%%%%%%%%%%%%%%%%%%%%%%%%%%%%%
% packages

\usepackage[utf8]{inputenc}
\usepackage[ngerman]{babel}
\usepackage[T1]{fontenc}

\usepackage{setspace}
\usepackage{ellipsis}
\usepackage{microtype}
\usepackage{lmodern}

\usepackage{lscape}
\usepackage{booktabs}			% \toprule, \midrule und \bottomrule in Tabellen
\usepackage{multirow}
\usepackage{paralist}

%\usepackage{scrhack} 
\usepackage{listings}
\lstset{
    language=C,
    breaklines=true,
    breakatwhitespace=true
    basicstyle=\footnotesize,
    numbers=left,
    numberstyle=\footnotesize,
    stepnumber=1,
    numbersep=5pt,
    extendedchars=true,
    inputencoding=utf8,
    breakindent=30pt,
    escapeinside={\%(}{\%)},
    captionpos=b
}

%\usepackage[pdftex]{graphicx} % Bereits von Beamer geladen
\graphicspath{{../Master_Thesis/images/}}
\usepackage{tikz}
\usepackage{wrapfig}
\usepackage{caption}         
\usepackage{subcaption}      
\usepackage{pgfgantt}        
\usepackage{rotating} 		

\hypersetup{
    pdftex,
    bookmarks, bookmarksopen, bookmarksopenlevel=1, bookmarksnumbered=true,
    pdfpagemode={UseNone},
    pdfpagelayout={SinglePage},
    plainpages=false,
    pdfkeywords={AUTOSAR},
    pdfsubject={AUTOSAR Einführung},
    pdftitle={AUTOSAR Einführung},
    pdfauthor={Martin Wichmann},
}



\newcommand{\inputImage}[1]{\input{../Master_Thesis/images/#1}}
% Bild einfügen:
%\centering
%\resizebox{0.3\linewidth}{!}{\inputImage{autosar_overview.dia}}

\newtranslation[to=ngerman]{Example}{Beispiel}

\usetheme{Warsaw}

\AtBeginSection[]
{
   \begin{frame}
        \frametitle{Inhaltsübersicht}
        \tableofcontents[currentsection,hideallsubsections]
   \end{frame}
}



%%%%%%%%%%%%%%%%%%%%%%%%%%%%%%%%%%%%%%%%%%%%%%%%%%%%
% Title
\author{Martin Wichmann}
\title{AUTOSAR Einführung}
\date{\today}
\institute{Ostfalia Hochschule für angewandte Wissenschaften}




%%%%%%%%%%%%%%%%%%%%%%%%%%%%%%%%%%%%%%%%%%%%%%%%%%%%
% begin document
\begin{document}

\begin{frame}
\maketitle
\end{frame}


\begin{frame}
\frametitle{Inhaltsübersicht}
\tableofcontents[hideallsubsections] % Einstellungen siehe Beamer User Guide Seite 99
\end{frame}





%%%%%%%%%%%%%%%%%%%%%%%%%%%%%%%%%%%%%%%%%%%%%%%%%%%%%%%%%%%%%%%%%%%5
% Einleitung
\section{Einleitung}
\label{sec:einleitung}

%%%%%%%%%%
\subsection{Allgemeines}
\label{sec:allgemeines}

%%%%%
\begin{frame}
\frametitle{Was ist AUTOSAR?}
    \begin{itemize}
        \item AUTomotive Open System ARchitecture
        \item Systemarchitektur
        \item Entwicklungsmodell
        \item Entwickelt von und für die Automobilindustrie
        \item "`Erbe"' von OSEK/VDX
        \item AUTOSAR ist nur Standard und beschreibt Schnittstellen
    \end{itemize}
\end{frame}

%%%%%
\begin{frame}
\frametitle{Warum AUTOSAR?}
    \begin{itemize}
        \item steigende Komplexität von E/E-Systemen
        \item "`Cooperate on standards, compete on implementation"'
        \item Ziele:
        \begin{itemize}
            \item Flexibilität
            \item Skalierbarkeit
            \item Qualität
            \item Zuverlässigkeit
        \end{itemize}
    \end{itemize}
\end{frame}

%%%%%
\begin{frame}
\frametitle{Versionshistorie}
    \begin{table}[h]
        \centering
        \begin{tabular}[h]{c c c l}
            \toprule
            Release & Datum & Phase & Notiz\\
            \midrule
                  & Juli 2003  &  & Gründung AUTOSAR\\
            \midrule
            1.0   & 08.07.2005 & \multirow{3}{*}{Phase I} & \\
            2.0	  & 04.05.2006 &  & Entwicklung des Standards\\
            2.1	  & 04.12.2006 &  & \\
            \midrule
            3.0	  & 21.12.2007 & \multirow{3}{*}{Phase II} & \\
            3.1	  & 15.08.2008 &  & Selektive Verbesserungen\\
            4.0	  & 18.12.2009 &  & \\
            \midrule
            4.0.2 & 15.04.2011 & \multirow{3}{*}{Phase III} & \\
            3.2	  & 13.05.2011 &  & Wartung\\
            4.0.3 & 22.12.2011 &  & \\
            \bottomrule
        \end{tabular}
        \label{tab:timeline}
    \end{table}
\end{frame}

%%%%%%%%%%
\subsection{Überblick}
\label{sec:überblick}

%%%%%
\begin{frame}[plain]
\frametitle{AUTOSAR Überblick}
    \begin{figure}[ht]
        \centering
        \resizebox{0.6\linewidth}{!}{\inputImage{autosar_overview.dia}}
        \label{fig:autosar_overview}
    \end{figure}
\end{frame}

%%%%%%%%%%
\subsection{Methdology}
\label{sec:methodology}

%%%%%
\begin{frame}
\frametitle{Stuktur}
    \begin{block}{Composition}
        \begin{itemize}
            \item "`Container"' für andere Kompositionen und SW-Cs
        \end{itemize}
    \end{block}
    \begin{block}{Software Component (SW-C)}
        \begin{itemize}
            \item "`Container"'
            \item Besitzen Ports
        \end{itemize}
    \end{block}
    \begin{block}{Runnable}
        \begin{itemize}
            \item Codesequenzen
            \item Durch Events (Timer, Daten,\dots) gestartet
            \item Senden und Empfangen via Ports
        \end{itemize}
    \end{block}
\end{frame}

%%%%%
\begin{frame}
\frametitle{VFB und RTE}
    \begin{block}{VFB}
        \begin{itemize}
            \item Virtual Functional Bus
            \item Systemsicht
            \item Abstrakte Beschreibung
        \end{itemize}
    \end{block}
    \begin{block}{RTE}
        \begin{itemize}
            \item Runtime Enviroment
            \item Realisiert VFB
            \item Pro ECU eine RTE
            \item Automatisch aus Systembeschreibung generiert
        \end{itemize}
    \end{block}
\end{frame}

%%%%%
\begin{frame}
\frametitle{Ports und Interfaces}
    \begin{block}{Port}
        \begin{itemize}
            \item Interaktionspunkte von Softwarekomponenten
            \item Provide Port (PPort)
            \item Require Port (RPort)
        \end{itemize}
    \end{block}
    \begin{block}{Interface}
        \begin{itemize}
            \item Beschreibt Port näher
            \item Daten/Operationen
            \item Port mit kompatiblen Interfaces können verbunden werden
        \end{itemize}
    \end{block}
\end{frame}

%%%%%%%%%%
\subsection{Notation}
\label{sec:Notation}

%%%%%
\begin{frame}
\frametitle{Notation}

    \begin{itemize}
        \item Arbeitsablauf durch SPEM
        \begin{itemize}
            \item \url{http://www.omg.org/spec/}
        \end{itemize}
        \item Systemmodelle durch eigene Notation
        \begin{itemize}
            \item Im Standard beschrieben unter: \url{http://autosar.org/download/R3.1/AUTOSAR_GraphicalNotation.pdf}
            \item Wird leider nicht durchgehend verwendet!
            \item (Noch unfertiges) Profil für Dia (Open-Source-Diagramm-Tool): \url{https://github.com/erebos42/dia_sheets}
        \end{itemize}
    \end{itemize}

\end{frame}

%%%%%
\begin{frame}
\frametitle{Komponenten Diagramm}
    \begin{figure}[ht]
        \centering
        \resizebox{\linewidth}{!}{\inputImage{SMLS_Modell.dia}}
        \label{fig:smls_modell}
    \end{figure}
\end{frame}





%%%%%%%%%%%%%%%%%%%%%%%%%%%%%%%%%%%%%%%%%%%%%%%%%%%%%%%%%%%%%%%%%%%5
% Einleitung
\section{Technik}
\label{sec:technik}

%%%%%
\begin{frame}
\frametitle{Technik}

\end{frame}

%%%%%%%%%%
\subsection{VFB/RTE}
\label{sec:VFB_RTE}
%%%%%
\begin{frame}
\frametitle{Technik}

\end{frame}

%%%%%%%%%%
\subsection{Ports und Interfaces}
\label{sec:ports_interfaces}
%%%%%
\begin{frame}
\frametitle{Technik}

\end{frame}

%%%%%%%%%%
\subsection{Basissoftware}
\label{sec:bsw}
%%%%%
\begin{frame}
\frametitle{Technik}

\end{frame}

%%%%%
\begin{frame}
\frametitle{Module}

\end{frame}





%%%%%%%%%%%%%%%%%%%%%%%%%%%%%%%%%%%%%%%%%%%%%%%%%%%%%%%%%%%%%%%%%%%5
% Anwendung an einem Beispiel
\section{Anwendung an einem Beispiel}
\label{sec:anwendung_beispiel}

%%%%%%%%%%
\subsection{Arbeitsablauf}
\label{sec:arbeitsablauf}

%%%%%
\begin{frame}
\frametitle{Entwicklungsprozess}
    \begin{figure}[ht]
        \centering
        \resizebox{0.98\linewidth}{!}{\inputImage{Autosar_Prozess.dia}}
        \label{fig:autosar_prozess}
    \end{figure}
\end{frame}

%%%%%%%%%%
\subsection{Beispiel}
\label{sec:beispiel}

%%%%%
\begin{frame}
\frametitle{Beispiel: System Description}
    \begin{figure}[ht]
        \centering
        \resizebox{\linewidth}{!}{\inputImage{SMLS_Modell.dia}}
        \label{fig:smls_modell2}
    \end{figure}
\end{frame}

%%%%%
\begin{frame}
\frametitle{Beispiel: Kommunikationsmatrix}

\end{frame}

%%%%%
\begin{frame}[fragile]
\frametitle{Beispiel: Applikationscode}
    \begin{verbatim}
void SWC_SMLS_R_SMLS (void) {
  Rte_Write_out_value(42);
}
    \end{verbatim}
\end{frame}

%%%%%
\begin{frame}
\frametitle{Beispiel: ECU-Konfiguration}

\end{frame}

%%%%%
\begin{frame}
\frametitle{Beispiel: Ergebnis}

\end{frame}









%%%%%%%%%%%%%%%%%%%%%%%%%%%%%%%%%%%%%%%%%%%%%%%%%%%%%%%%%%%%%%%%%%%5
% Literaturangaben
\appendix
\section*{Literatur}
\label{sec:Literatur}

\begin{frame}


\begin{thebibliography}{10}

\bibitem[1]{1} \textsc{Olaf Kindel, Mario Driedrich}: {\em Softwareentwicklung mit AUTOSAR: Grundlagen, Engineering, Management in der Praxis.} dpunkt.verlag, 2009.

\bibitem[2]{2} \textsc{Peter Löw, Roland Pabst, Erwin Petry}: {\em Funktionale Sicherheit in der Praxis.} dpunkt.verlag, 2010.

\bibitem[3]{3} \textsc{AUTOSAR}: {\em Technical Overview.} Online unter: \url{http://autosar.org/download/R3.1/AUTOSAR_TechnicalOverview.pdf}

\bibitem[4]{4} \textsc{AUTOSAR}: {\em Layered Software Architecture.} Online unter: \url{http://autosar.org/download/R3.1/AUTOSAR_LayeredSoftwareArchitecture.pdf}

\end{thebibliography}


\end{frame}

\end{document}

