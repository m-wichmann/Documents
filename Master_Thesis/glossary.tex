\newglossaryentry{glo:nm}
 {
   name=Netzwerkmanagement,
   description={Verfahren, um einen synchronen Zustandsübergang in einem Bussystem zu ermöglichen. Wird oft auch mit NM abgekürzt}
}


\newglossaryentry{glo:port}
 {
   name=Port,
   description={Kommunikations-Schnittstelle, die durch ein Interface näher beschrieben wird}
}

\newglossaryentry{glo:interface}
 {
   name=Interface,
   description={Konkrete Beschreibung eines Ports. Beinhaltet Datenelemente und gibt die Richtung der Kommunikation an (provides und requires)}
}

\newglossaryentry{glo:swc}
 {
   name=Software Component,
   description={Kann eine Composition oder eine atomare Softwarekomponente sein. Compositions enthalt weitere Compositions und atomare Softwarekomponenten. Eine atomare Softwarekomponenten enthält eine Implementation und damit Runnables. Kommunikation wird via Ports realisiert}
}

\newglossaryentry{glo:comp}
 {
   name=Composition,
   description={Fasst andere Compositions und Softwarekomponenten zusammen}
}

\newglossaryentry{glo:runnable}
 {
   name=Runnable,
   description={Ausführbarer Teil einer Software Komponente. Wird durch RTE Events aufgerufen und hat Zugriff auf konfigurierte Interfaces}
}

\newglossaryentry{glo:bsw}
 {
   name=Basis-Software,
   description={Eigentliches Betriebssystem von AUTOSAR. Liegt zwischen Hardware und RTE und ist in Module eingeteilt}
}

\newglossaryentry{glo:rte}
 {
   name=Runtime-Enviroment,
   description={Realisiert konkrete Implementierung des VFB und stellt damit Kommunikationsverbindungen zwischen Softwarekomponenten und der Basis-Software bereit}
}

\newglossaryentry{glo:irv}
 {
   name=Interrunnable-Variable,
   description={Variable, die durch mehrere Runnables verwendet werden kann. Zugriff muss im Modell festgelegt werden}
}

\newglossaryentry{glo:vfb}
 {
   name=Virtual Function Bus,
   description={Abstraktes Modell der Kommunikation und der Softwarekomponenten. Wird konkret durch eine oder mehrere RTEs realisiert}
}

\newglossaryentry{glo:host}
 {
   name=Host,
   description={Der Host ist bei der Virtualisierung das System, dass den Hypervisor ausführt}
}

\newglossaryentry{glo:gast}
 {
   name=Gast,
   description={System, das virtuell auf dem Hypervisor ausgeführt wird}
}
