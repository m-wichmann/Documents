% TODO:
% - 

%%%%%%%%%%%%%%%%%%%%%%%%%%%%%%%%%%%%%%%%%%%%%%%%%%%%
% documentclss
\documentclass[]{beamer}
%\documentclass[handout]{beamer} %Drucker Version


%%%%%%%%%%%%%%%%%%%%%%%%%%%%%%%%%%%%%%%%%%%%%%%%%%%%
% packages

\usepackage[utf8]{inputenc}
\usepackage[ngerman]{babel}
\usepackage[T1]{fontenc}

\usepackage{setspace}
\usepackage{ellipsis}
\usepackage{microtype}
\usepackage{lmodern}

\usepackage{listings}
\lstset{
    language=C,
    breaklines=true,
    breakatwhitespace=true
    basicstyle=\footnotesize,
    numbers=left,
    numberstyle=\footnotesize,
    stepnumber=1,
    numbersep=5pt,
    extendedchars=true,
    inputencoding=utf8,
    breakindent=30pt,
    escapeinside={\%(}{\%)},
    captionpos=b
}

\usepackage{graphicx}
\graphicspath{{../Master_Thesis/images/}}
\usepackage{tikz}

\hypersetup{
    pdftex,
    bookmarks, bookmarksopen, bookmarksopenlevel=1, bookmarksnumbered=true,
    pdfpagemode={UseNone},
    pdfpagelayout={SinglePage},
    plainpages=false,
    pdfkeywords={AUTOSAR, Virtualisierung, ECU, Python, CAN},
    pdfsubject={Virtualisierung von AUTOSAR Softwarekomponenten für die Erprobung},
    pdftitle={Virtualisierung von AUTOSAR Softwarekomponenten für die Erprobung},
    pdfauthor={Martin Wichmann},
}

\usepackage{booktabs}
\usepackage{multirow}


\newcommand{\inputImage}[1]{\input{../Master_Thesis/images/#1}}
% Bild einfügen:
%\centering
%\resizebox{0.3\linewidth}{!}{\inputImage{autosar_overview.dia}}

\newtranslation[to=ngerman]{Example}{Beispiel}

\usetheme{Warsaw}

\AtBeginSection[]
{
   \begin{frame}
        \frametitle{Inhaltsübersicht}
        \tableofcontents[currentsection,currentsubsection]
   \end{frame}
}



%%%%%%%%%%%%%%%%%%%%%%%%%%%%%%%%%%%%%%%%%%%%%%%%%%%%
% Title
\author{Martin Wichmann}
\title[Virtualisierung von AUTOSAR Softwarekomponenten]{Virtualisierung von AUTOSAR Softwarekomponenten für die Erprobung}
\date{\today}
\institute{Ostfalia Hochschule für angewandte Wissenschaften}




%%%%%%%%%%%%%%%%%%%%%%%%%%%%%%%%%%%%%%%%%%%%%%%%%%%%
% begin document
\begin{document}

\begin{frame}
\maketitle
\end{frame}


\begin{frame}
\frametitle{Inhaltsübersicht}
\tableofcontents
\end{frame}





%%%%%%%%%%%%%%%%%%%%%%%%%%%%%%%%%%%%%%%%%%%%%%%%%%%%%%%%%%%%%%%%%%%5
% Einleitung
\section{Einleitung}
\label{sec:einleitung}

\begin{frame}
\frametitle{Frametitle}



\end{frame}

%%%%%%%%%%%%%%%%%%%%%%%%%%%%%%%%%%%%%%%%%%%%%%%%%%%%%%%%%%%%%%%%%%%5
% Fazit
\section{Fazit}
\label{sec:Fazit}

\begin{frame}
\frametitle{Fazit}


\end{frame}








%%%%%%%%%%%%%%%%%%%%%%%%%%%%%%%%%%%%%%%%%%%%%%%%%%%%%%%%%%%%%%%%%%%5
% Literaturangaben
\appendix
\section*{Literatur}
\label{sec:Literatur}

\begin{frame}


\begin{thebibliography}{10}

\bibitem[1]{1} \textsc{Olaf Kindel, Mario Driedrich}: {\em Softwareentwicklung mit AUTOSAR: Grundlagen, Engineering, Management in der Praxis.} dpunkt.verlag, 2009.

\bibitem[2]{2} \textsc{Peter Löw, Roland Pabst, Erwin Petry}: {\em Funktionale Sicherheit in der Praxis.} dpunkt.verlag, 2010.

\bibitem[3]{3} \textsc{AUTOSAR}: {\em Technical Overview.} Online unter: \url{http://autosar.org/download/R3.1/AUTOSAR_TechnicalOverview.pdf}

\bibitem[4]{4} \textsc{AUTOSAR}: {\em Layered Software Architecture.} Online unter: \url{http://autosar.org/download/R3.1/AUTOSAR_LayeredSoftwareArchitecture.pdf}

\end{thebibliography}


\end{frame}

\end{document}

